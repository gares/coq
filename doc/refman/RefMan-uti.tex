\chapter[Utilities]{Utilities\label{Utilities}}

The distribution provides utilities to simplify some tedious works
beside proof development, tactics writing or documentation.

\section[Building a toplevel extended with user tactics]{Building a toplevel extended with user tactics\label{Coqmktop}\ttindex{coqmktop}}

The native-code version of \Coq\ cannot dynamically load user tactics
using {\ocaml} code. It is possible to build a toplevel of \Coq,
with {\ocaml} code statically linked, with the tool {\tt
  coqmktop}.

For example, one can build a native-code \Coq\ toplevel extended with a tactic
which source is in {\tt tactic.ml} with the command
\begin{verbatim}
     % coqmktop -opt -o mytop.out tactic.cmx
\end{verbatim}
where {\tt tactic.ml} has been compiled with the native-code
compiler {\tt ocamlopt}. This command generates an executable
called {\tt mytop.out}. To use this executable to compile your \Coq\
files, use {\tt coqc -image mytop.out}.

A basic example is the native-code version of \Coq\ ({\tt coqtop.opt}),
which can be generated by {\tt coqmktop -opt -o coqopt.opt}.


\paragraph[Application: how to use the {\ocaml} debugger with Coq.]{Application: how to use the {\ocaml} debugger with Coq.\index{Debugger}}

One useful application of \texttt{coqmktop} is to build a \Coq\ toplevel in
order to debug your tactics with the {\ocaml} debugger.
You need to have configured and compiled \Coq\ for debugging
(see the file \texttt{INSTALL} included in the distribution).
Then, you must compile the Caml modules of your tactic with the
option \texttt{-g} (with the bytecode compiler) and build a stand-alone
bytecode toplevel with the following command:

\begin{quotation}
\texttt{\% coqmktop -g -o coq-debug}~\emph{<your \texttt{.cmo} files>}
\end{quotation}


To launch the \ocaml\ debugger with the image you need to execute it in
an environment which correctly sets the \texttt{COQLIB} variable.
Moreover, you have to indicate the directories in which
\texttt{ocamldebug} should search for Caml modules.

A possible solution is to use a wrapper around \texttt{ocamldebug}
which detects the executables containing the word \texttt{coq}. In
this case, the debugger is called with the required additional
arguments. In other cases, the debugger is simply called without additional
arguments. Such a wrapper can be found in the \texttt{dev/}
subdirectory of the sources.

\section[Building a \Coq\ project with {\tt coq\_makefile}]
{Building a \Coq\ project with {\tt coq\_makefile}
\label{Makefile}
\ttindex{Makefile}
\ttindex{coq\_Makefile}
\ttindex{\_CoqProject}}

The majority of \Coq\ projects are very similar: a collection of {\tt .v}
files and eventually some {\tt .ml} ones (a \Coq\ plugin).  The main piece
of metadata needed in order to build the project are the command 
line options to {\tt coqc} (e.g. {\tt -R, -I},
\SeeAlso Section~\ref{coqoptions}). Collecting the list of files and
options is the job of the {\tt \_CoqProject} file.

A simple example of a {\tt \_CoqProject} file follows:

\begin{verbatim}
-R theories/ MyCode
theories/foo.v
theories/bar.v
-I src/
src/baz.ml4
src/bazaux.ml
src/qux_plugin.mlpack
\end{verbatim}

Currently, both \CoqIDE{} and Proof General (version $\geq$ 4.3pre) understand
{\tt \_CoqProject} files and invoke \Coq\ with the desired options.

The {\tt coq\_makefile} utility can be used to set up a build infrastructure
for the \Coq\ project based on makefiles.  The recommended way of
invoking {\tt coq\_makefile} is the following one:

\begin{verbatim}
coq_makefile -f _CoqProject -o CoqMakefile
\end{verbatim}

Such command generates the following files:
\begin{description}
	\item[{\tt CoqMakefile}] is a generic makefile for GNU Make that provides targets to build the project (both {\tt .v} and {\tt .ml*} files), to install it system-wide in the {\tt coq-contrib} directory (i.e. where \Coq\ is installed) as well as to invoke {\tt coqdoc} to generate html documentation.

	\item[{\tt CoqMakefile.conf}] contains make variables assignments that reflect the contents of the {\tt \_CoqProject} file as well as the path relevant to \Coq{}.
\end{description}

An optional file {\bf {\tt CoqMakefile.local}} can be provided by the user in order to extend {\tt CoqMakefile}.  In particular one can declare custom actions to be performed before or after the build process. Similarly one can customize the install target or even provide new targets.  Extension points are documented in the {\tt CoqMakefile} file.

The extensions of the files listed in {\tt \_CoqProject} is
used in order to decide how to build them  In particular:

\begin{itemize}
\item {\Coq} files must use the \texttt{.v} extension
\item {\ocaml} files must use the \texttt{.ml} or \texttt{.mli} extension
\item {\ocaml} files that require pre processing for syntax extensions (like {\tt VERNAC EXTEND}) must use the \texttt{.ml4} extension
\item In order to generate a plugin one has to list all {\ocaml} modules (i.e. ``Baz'' for ``baz.ml'') in a \texttt{.mlpack} file (or \texttt{.mllib} file).
\end{itemize}

The use of \texttt{.mlpack} files has to be preferred over \texttt{.mllib}
files, since it results in a ``packed'' plugin: All auxiliary 
modules (as {\tt Baz} and {\tt Bazaux}) are hidden inside
the plugin's ``name space'' ({\tt Qux\_plugin}).
This reduces the chances of begin unable to load two distinct plugins
because of a clash in their auxiliary module names.

\paragraph{Notes about including the generated Makefile}

This practice is discouraged. The contents of this file, including variable names
and status of rules shall change in the future.  Users are advised to
include {\tt Makefile.conf} or call a target of the generated Makefile
as in {\tt make -f Makefile target} from another Makefile.

\paragraph{Notes for users of {\tt coq\_makefile} with version $<$ 8.7}

\begin{itemize}
\item Support for ``sub-directory'' is deprecated.  To perform actions before
	or after the build (like invoking make on a subdirectory) one can
	hook in {\tt pre-all} and {\tt post-all} extension points
\item \texttt{-extra-phony} and \texttt{-extra} are deprecated.  To provide
	additional target ({\tt .PHONY} or not) please use 
	{\tt CoqMakefile.local}
\end{itemize}

\section[Modules dependencies]{Modules dependencies\label{Dependencies}\index{Dependencies}
  \ttindex{coqdep}}

In order to compute modules dependencies (so to use {\tt make}),
\Coq\ comes with an appropriate tool, {\tt coqdep}.

{\tt coqdep} computes inter-module dependencies for \Coq\ and
\ocaml\ programs, and prints the dependencies on the standard
output in a format readable by make.  When a directory is given as
argument, it is recursively looked at.

Dependencies of \Coq\ modules are computed by looking at {\tt Require}
commands ({\tt Require}, {\tt Requi\-re Export}, {\tt Require Import},
but also at the command {\tt Declare ML Module}.

Dependencies of \ocaml\ modules are computed by looking at
\verb!open! commands and the dot notation {\em module.value}. However,
this is done approximately and you are advised to use {\tt ocamldep}
instead for the \ocaml\ modules dependencies.

See the man page of {\tt coqdep} for more details and options.

The build infrastructure generated by {\tt coq\_makefile}
uses {\tt coqdep} to automatically compute the dependencies
among the files part of the project.

\section[Documenting \Coq\ files with coqdoc]{Documenting \Coq\ files with coqdoc\label{coqdoc}
\ttindex{coqdoc}}

\input{./coqdoc}

\section[Embedded \Coq\ phrases inside \LaTeX\ documents]{Embedded \Coq\ phrases inside \LaTeX\ documents\label{Latex}
  \ttindex{coq-tex}\index{Latex@{\LaTeX}}}

When writing a documentation about a proof development, one may want
to insert \Coq\ phrases inside a \LaTeX\ document, possibly together with
the corresponding answers of the system. We provide a
mechanical way to process such \Coq\ phrases embedded in \LaTeX\ files: the
{\tt coq-tex} filter.  This filter extracts Coq phrases embedded in
LaTeX files, evaluates them, and insert the outcome of the evaluation
after each phrase.

Starting with a file {\em file}{\tt.tex} containing \Coq\ phrases,
the {\tt coq-tex} filter produces a file named {\em file}{\tt.v.tex} with
the \Coq\ outcome.

There are options to produce the \Coq\ parts in smaller font, italic,
between horizontal rules, etc.
See the man page of {\tt coq-tex} for more details.

\medskip\noindent {\bf Remark.} This Reference Manual and the Tutorial
have been completely produced with {\tt coq-tex}.


\section[\Coq\ and \emacs]{\Coq\ and \emacs\label{Emacs}\index{Emacs}}

\subsection{The \Coq\ Emacs mode}

\Coq\ comes with a Major mode for \emacs, {\tt gallina.el}. This mode provides
syntax highlighting
and also a rudimentary indentation facility
in the style of the Caml \emacs\ mode.

Add the following lines to your \verb!.emacs! file:

\begin{verbatim}
  (setq auto-mode-alist (cons '("\\.v$" . coq-mode) auto-mode-alist))
  (autoload 'coq-mode "gallina" "Major mode for editing Coq vernacular." t)
\end{verbatim}

The \Coq\ major mode is triggered by visiting a file with extension {\tt .v},
or manually with the command \verb!M-x coq-mode!.
It gives you the correct syntax table for
the \Coq\ language, and also a rudimentary indentation facility:
\begin{itemize}
  \item pressing {\sc Tab} at the beginning of a line indents the line like
    the line above;

  \item extra {\sc Tab}s increase the indentation level
    (by 2 spaces by default);

  \item M-{\sc Tab} decreases the indentation level.
\end{itemize}

An inferior mode to run \Coq\ under Emacs, by Marco Maggesi, is also
included in the distribution, in file \texttt{coq-inferior.el}.
Instructions to use it are contained in this file.

\subsection[{\ProofGeneral}]{{\ProofGeneral}\index{Proof General@{\ProofGeneral}}}

{\ProofGeneral} is a generic interface for proof assistants based on
Emacs. The main idea is that the \Coq\ commands you are
editing are sent to a \Coq\ toplevel running behind Emacs and the
answers of the system automatically inserted into other Emacs buffers.
Thus you don't need to copy-paste the \Coq\ material from your files
to the \Coq\ toplevel or conversely from the \Coq\ toplevel to some
files.

{\ProofGeneral} is developed and distributed independently of the
system \Coq. It is freely available at \verb!https://proofgeneral.github.io/!.


\section[Module specification]{Module specification\label{gallina}\ttindex{gallina}}

Given a \Coq\ vernacular file, the {\tt gallina} filter extracts its
specification (inductive types declarations, definitions, type of
lemmas and theorems), removing the proofs parts of the file. The \Coq\
file {\em file}{\tt.v} gives birth to the specification file
{\em file}{\tt.g} (where the suffix {\tt.g} stands for \gallina).

See the man page of {\tt gallina} for more details and options.


\section[Man pages]{Man pages\label{ManPages}\index{Man pages}}

There are man pages for the commands {\tt coqdep}, {\tt gallina} and
{\tt coq-tex}. Man pages are installed at installation time
(see installation instructions in file {\tt INSTALL}, step 6).

%BEGIN LATEX
\RefManCutCommand{ENDREFMAN=\thepage}
%END LATEX

%%% Local Variables:
%%% mode: latex
%%% TeX-master: t
%%% End:
